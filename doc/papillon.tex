%%%
%
% Papillon Documentation
% $Id: papillon.tex,v 1.81 2004/03/30 16:56:35 kr Exp $
% 
%%%

\documentclass[a4paper,12pt,headsepline,normalheadings,bibtotoc
               pointlessnumbers]{scrreprt}       

\usepackage{mydefs}	% Eigene Definitionen und zu ladende Pakete
\usepackage{mylayout}	% Layout des Dokuments, der �berschriften, etc...

\begin{document}
   
   \thispagestyle{empty}

   \vspace*{\stretch{1}}
   \HRule
   
   \begin{center}
   \LARGE \bf \sffamily 
       %\includegraphics[width=0.4\textwidth]{papillon-logo}\\
       Papillon \version -- Solaris Security Module \\
       Documentation and Manual
   \HRule
   \end{center} 
   
   \begin{center}
   \vspace*{\stretch{4}}

   \upshape \large Konrad Rieck {\tt (kr@roqe.org)} \\[12pt]
            \normalsize
	     $~$ \today $~-~$Revision: 1.81 $~$\\
	     Copyright 2000 -- 2003, 2006 

   \vspace*{\stretch{1}}

   \end{center}

\newpage

\tableofcontents
\setlength{\parskip}{\myparskip}

%%
% Chapter 1
%%
\chapter{Introduction}


Papillon is a security module designed for the Solaris Operating
Environment~\cite{SOE}. It provides security mechanisms and
protections that improve the overall security of the system by adding
new functionality to the kernel. The security mechanisms and
protections have been inspired by the Openwall~\cite{OW} and the
HAP~\cite{HAP} Linux kernel patches and address common Unix security
issues that are also present in the Solaris Operating Environment.  In
the current version Papillon supports Solaris (8, 9 and 10) and
OpenSolaris~\cite{OS} running on x86 and SPARC architectures in
32 or 64bit mode.\\

Papillon follows the philosophy of {\em prevention through
  restriction}. By adding minimal restrictions to resources, such as
symbolic links or FIFOs, compromises based on common attack techniques
can be prevented without influence on the system's usability. Papillon
is a great addition to already existing security solutions such as
Solaris DTrace, the Basic Security Module (BSM) and the non-executable
stack on the Solaris SPARC Edition. All features and protections of
the Papillon module can be en- and disabled at compilation time or
even at runtime, therefore the functionality can be optimally
adapted to a specific system.\\

Even though there have been several requests, the module is not designed to
weaken the super-user's privileges and protect against intruders that
already compromised the system. Restricting these privileges requires a
special design of the operating system, e.g. in Trusted Solaris~\cite{TSOE}
or OpenBSD~\cite{OBSD}, and is nearly impossible to implement using plain
loadable kernel modules.\\ 

The following documentation and manual is divided into two major parts. The
first part introduces Papillon's functionality and its impact on the
system's security in detail. The second part describes the compilation,
installation and configuration of the Papillon module and its components. In
order to correctly integrate the module into your system, it is essential to
study both parts of this documentation, in order to learn about the provided
functionality and its installation. Papillon is not one of those simple 
software packages, incorrect installation or misconfiguration may have an
opposite effect on the security of the target system.

%%
% Chapter 2
%%
\chapter{Papillon's functionality}

For simplicity Papillon's functionality has been divided into functional
units that act either as so called {\em features} or security 
{\em protections}. 

\setlength{\parskip}{\tocparskip}
\begin{itemize}
   \item {\bf Features}\\
         Features add completely new functionality to the kernel. They can
         be switched {\em on} or {\em off} either at compilation time or
	 later at runtime using the provided control tool {\code papctl}. 

   \item {\bf Protections}\\ 
	 Protections restrict access to resources if specific conditions
	 occur.  A protection has a behaviour that can be {\em none} for doing
	 nothing, {\em warn} for warning only or {\em deny} for warning and
	 denying access to the specific resource.
\end{itemize}	 
\setlength{\parskip}{\myparskip}

\section{Features}

Following is a description of each feature implemented in the Papillon
module \version. For each feature the following part lists a short
introduction to the addressed security issue and a description how the
module fixes the problem. Additionally some implementation details are
provided that are intended for developers that want to modify or extend the
module's code.\\

As mentioned above features, loaded with the Papillon module into the
Solaris kernel, can be switched {\em on} or {\em off} either at compilation
time or later at runtime. 

\subsection{Restricted Proc}

By default users in the Solaris Operating Environment are able to monitor
all active processes (e.g. by using the programs {\code ps} or {\code top}).
An attacker that has local access to the system might gather useful
information by watching system daemons and other users' processes. The
public information about all running processes also represents a lack of
privacy, if a system hosts several users.\\

If the Restricted Proc feature is enabled, users are only able to view own
processes which are running under their user ID (UID). It is impossible for
an attacker to monitor other users' processes because Papillon directly
restricts access to the {\code /proc} file system which is the global source
for all information about running processes.\\

In order to allow system's maintenance the super-user is able to view all
processes. A special group can be added whose members are also able to view
running processes when the Restricted Proc feature is enabled.\\

Papillon extends the {\code access()} function of the procfs vnode
operations provided by {\code prvnodeops} in order to implement the above
feature. The Solaris OE does set the correct permissions on the files inside
the {\code /proc} file system but does not implement an {\code access()}
function in the procfs kernel module. Papillon simply adds this missing
{\code access()} function. The Restricted Proc feature has been inspired by
the Openwall~\cite{OW} Linux  kernel patch and is also briefly discussed in
the presentation~\cite{SKP}  by Konrad Rieck and Job de Haas.

\subsection{Pseudo Promiscuous Flag}

The Solaris Operating Environment 9 and previous Solaris versions don't
provide a promiscuous mode flag for network adapters that is exported to the
user. An administrator is not able to monitor a network device for an
attacker sniffing on the device.\\

Papillon is able to log all attempts to turn a network device into
promiscuous mode that are done using the DLPI interface. Most sniffers, e.g.
{\code snoop} or libpcap-based sniffers as {\code tcpdump}, use that 
interface to communicate with network adapters. Requests that are
performed using a different approach are not detected. Below is an example
syslog entry created by the Papillon module indicating that a network device
has been put into promiscuous mode 

\bcode 
\begin{verbatim} 
Mar 26 20:16:37 fluffy papillon: WARNING: Promiscuous mode enabled on
interface hme (cmd: tcpdump, pid: 6179, uid: 0, gid: 1). 
\end{verbatim} 
\ecode

Papillon intercepts the {\code putmsg()} system call and filters messages
that match DLPI {\em promiscuous on} requests. If such a message is detected
a warning is send to the syslog. The module is not able to detect a network
interface changing back from promiscuous mode to normal operation mode.

\subsection{Secure STDIO File Descriptors}

By default Unix uses the file descriptors 0, 1 and 2 for special
purposes -- the standard IO (STDIO) which includes in- and output of
programs to and from the console. Typically these in- and outputs can be
combined and filtered using so called pipes. Below is a list of the three
STDIO file descriptors.

\begin{center}
\begin{tabular}{r | l}
 STDIN & File descriptor 0 is used for the standard input stream. \\
 STDOUT & File descriptor 1 is used for the standard output stream. \\
 STDERR & File descriptor 2 is used for the standard error stream. \\
\end{tabular}
\end{center}

If an attacker closes one of these file descriptors and executes an program
with the setuid or setgid bit set, a file descriptor inside the program
might be assigned to one of the closed STDIO file descriptors. In this case
information written to STDIN, STDOUT or STDERR might be written to a file.
By using this technique an attacker is able to destroy or modify system
files.\\

Papillon intercepts the execution of all binaries that have the setuid or
setgid bit set. If one of the STDIO file descriptors is closed before
executing such a program, Papillon fake opens the descriptor during the
execution of the program. Therefore no program with the setuid or setgid bit
set is able to accidentally assign a file to the STDIO file descriptors.
Below is an example entry from the syslog that illustrates the fake opening
of STDIO file descriptors.

\bcode
\begin{verbatim}
Mar 26 20:25:47 fluffy papillon: WARNING: Fake opening STDERR before
executing /tmp/a (cmd: sh -c /tmp/a, pid: 6472, uid: 101, gid: 101).
\end{verbatim}
\ecode

Papillon intercepts the {\code execve()} system call and watches vnodes with
the setuid or setgid bit set. If one of the STDIO file descriptors is closed
before execution, it is faked opened using the kernel allocation routine
{\code ualloc()} and unallocated after execution. The problem of closed
STDIO file descriptors is discussed inside the Openwall~\cite{OW} Linux
kernel patch.

\subsection{Module Hiding}

In most cases it is not necessary to hide a security module. But if an
administrator wants to monitor an existing attacker, it might be necessary
to make the attacker believe that the system is not protected by any
security software. \\ 

Papillon is able to remove itself from the list of loaded kernel modules and
can operate invisible. It also denies any access to the module's files and
hides them from directory listings, including the module itself,
init-scripts and the control program. The super-user is able to view and
access all of these files. The list of files that are hidden can be extended
at compilation time, but not at runtime, so that an administrator can add
other files that are not visible to the system's users.\\

Papillon unlinks itself from the list of loaded modules and relinks itself
back in if requested. The module intercepts the {\code vop\_lookup()} and
{\code vop\_readdir()} functions from the root file system in order to hide
files from direct access and directory listings. The file hiding mechanism
is based on Job de Haas' kernel module. Directory entries are removed by
patching the length of previous {\code dirent64} entry using {\code d\_len}.

\section{Protections}

Following is a description of each protection integrated in the Papillon
module \version. Similar to the previous section for each protection an
addressed security issue is introduced and a detailed description of the
protection given. Additionally some implementation details are provided that
are intended for developers that want to modify or extend the module's
code.\\

Protections restrict access to resources (e.g. access to file) if specific
conditions occur. A protection has a behaviour that can be {\em none} for
doing nothing, {\em warn} for warning only or {\em deny} for warning and
denying access to the resource.

\subsection{Symbolic Link Protection}

Directories with the sticky bit (octal mode 1000) and write-all permissions
have a specific behaviour: files created in these directories can only be
removed by the file owner or the super-user even though write permissions
are granted to all users. A typical example for such a directory is the
{\code /tmp} directory. An attacker can use this specific behaviour to
perform a symbolic link attack which is based on a symbolic link that
redirects output from a temporary file.\\

Papillon provides a simple symbolic link protection based on the
Openwall~\cite{OW} Linux kernel patch. If a user wants to follow a symbolic
link that is within a directory with the sticky bit set, access is denied if
all of the following conditions are true:

\begin{itemize}
\item[+] The parent directory of the symbolic link has the sticky bit set.
\item[+] The parent directory of the symbolic link has a different owner
         than the symbolic link.
\item[+] The symbolic link is not owned by the user who is accessing it.
\end{itemize}

Due to this protection an attacker is not able to increase his privileges if
the super-user executes a binary that uses temporary files which are
vulnerable to a symbolic link attack. Following is a syslog entry
illustrating the output of Papillon if opening a symbolic link is denied by 
the module. Instead of denying access, the administrator can also set this
protection to warning only mode.

\bcode
\begin{verbatim}
Mar 26 20:25:47 fluffy papillon: WARNING: Denied following symlink
/tmp/a (cmd: cat /tmp/a, pid: 6448, uid: 0, gid: 1).
\end{verbatim}
\ecode

Papillon watches all calls to the {\code open()} and {\code open64()} system
calls. If a symbolic link should be opened that is placed in a directory
with the sticky bit  and the above conditions match, the open fails with
permission denied ({\code EPERM}).

\subsection{Hard Link Protection}

An attacker can perform most symbolic link attacks by using hard links. If
the symbolic links are protected, it is likely that hard links will be used
in future exploits. Therefore it is necessary to bundle a Hard Link
Protection with the existing Symbolic Link Protection of Papillon.\\ 

There is also another common problem with hard links in the Solaris
Operating Environment. Users are able to create hard links to file which
they don't own. After the creation of such a hard link the user is not able
to delete the created link.\\

Papillon fixes both problems. If the hard link protection is enabled users
can not create hard links to files which they don't own. The super-user is
able to create hard links to all files. Following is an example entry from
the syslog.

\bcode
\begin{verbatim}
Mar 26 20:25:47 fluffy papillon: WARNING: Denied creating hardlink
from a to b (cmd: ln a b, pid: 6443, uid: 101, gid: 101).
\end{verbatim}
\ecode

The {\code link()} system call is intercepted to implement above protection.
The module returns permission denied ({\code EPERM}) if a user tries to create
a hard link to a file which he doesn't own.

\subsection{FIFO Protection}

The so called FIFO special file is used to queue input using a first-in 
first-out algorithm, it can be created using the command {\code mkfifo}. If
the sticky bit is set and write-all permissions are granted to a directory,
an attacker is able open a FIFO special file inside this directory with the 
{\code O\_CREAT} flag and clear all the content stored in the FIFO queue.\\

Papillon implements a FIFO Protection that prevents a FIFO special file from
loosing its content if an attacker tries to open it and the following
conditions occur. For simplicity the term FIFO is used instead of FIFO
special file.

\begin{enumerate}
\item[+] The parent directory of the FIFO has the sticky bit set.
\item[+] The parent directory of the FIFO  has a different owner 
         than FIFO. 
\item[+] The {\code O\_CREAT} flag is set in the opening mode.
\item[+] The FIFO is not owned by the user who is accessing it.
\item[+] The user is not the super-user.
\end{enumerate}

By adding this restriction to the opening of FIFOs all attacks based on  the
removal of FIFO special file content can be prevented. Below is an example
entry from the syslog that illustrates a denied open request for a FIFO
special file.

\bcode
\begin{verbatim}
Mar 26 20:25:47 fluffy papillon: WARNING: Denied opening FIFO 
/tmp/a (cmd: ./fifoattack /tmp/a, pid: 6453, uid: 101, gid: 101).
\end{verbatim}
\ecode

Papillon watches all calls to the {\code open()} and {\code open64()} system
calls, if a FIFO is to be opened with the {\code O\_CREAT} flag in a
directory with the sticky bit, access is denied if the above conditions
match. In this case Papillon returns permission denied ({\code EPERM}). This
protection is also based on the Openwall~\cite{OW} Linux kernel patch. 

\subsection{Chroot Protection}

The {\code chroot()} system call is often used to create another security
layer between an application and the operating systems, but it has been
initially designed as a safe (not secure) installation environment. An
attacker that gained super-user privileges in a chroot environment will
focus on removing the chroot restrictions.\\

There are several methods for breaking out of a chroot environment, common 
techniques include: re-calling the {\code chroot()} system call to set a 
new file system root, mounting an outside resource into the chroot
environment, creating block or character devices to outside resources from 
inside the chroot environment, loading malicious kernel modules and changing
permissions on administration binaries inside the chroot environment.\\

Based on the HAP~\cite{HAP} Linux kernel patch Papillon prevents these
attacks if a process runs inside a chroot environment. Several system calls
will restrict access when called from a chroot environment, e.g. the system
call for creating block or character devices. Below is an example entry from
the syslog showing a denied device creation request inside a chroot
environment.

\bcode
\begin{verbatim}
Mar 26 20:25:48 fluffy papillon: WARNING: Denied creating device
node /dev_null chroot'ed (cmd: ./break, pid: 6481, uid: 0, gid: 1).
\end{verbatim}
\ecode

Papillon intercepts the following system calls and restricts access if the
running process has the chroot vnode set ({\code u.u\_rdir}):  {\code
chroot()}, {\code mount()}, {\code mknod()},  {\code xmknod()},  {\code
modctl()}  and {\code chmod()}. The system calls return {\code EPERM} 
called from inside a chroot environment. The initial version of this 
protection has been implemented by Heiko Krupp / MIP GmbH.

\subsection{Setuid Execution Protection}

A lot of vulnerabilities that allow a local attacker to change his
privileges exploit bugs in setuid or setgid binaries. Usually the attacker
executes a shell or another program from within the setuid or setgid binary
to gain more privileges. \\

The Setuid Execution Protection monitors the execution of programs on the
system and is activated whenever a program with the setuid or setgid bit
executes a child program.  The protection can be used to simply log the
execution of these child programs or might also be used to deny any
execution of child programs from within setuid or setgid programs (which
might be too restrictive). Before Papillon is compiled, the white-list of
programs that don't pass this protection can be extended with programs that
are known to be secure. An example output from the syslog is listed below.

\bcode
\begin{verbatim}
Mar 26 19:01:31 fluffy papillon: WARNING: Executing /tmp/a by 
setuid parent /tmp/b (cmd: /tmp/b, pid: 5039, uid: 101, gid: 10).
\end{verbatim}
\ecode

Papillon intercepts the {\code execve()} system call to monitor the
execution of programs and their parent processes. The {\code p_exec} entry
is used to retrieve the parent process' vnode.

%%%
% Chapter 3
%%%

\chapter{Installation and Configuration}

\section{Installation}

Papillon can be installed from source or binary packages which are available
at the Papillon website~\cite{PAP} (\www{www.roqe.org/papillon}). In general
it is recommended to compile and install the module from a source package,
even though the installation of binary package has become a practical and
common method. Compiling the Papillon module allows better code optimisation
and performance due to the target system's C compiler and corresponding
compiler options, and also supports compilation time configuration which can
be used to minimise the module's size and set default values to optimally 
adapt to the target system.\\

The following section describes the compilation and installation process
including details on compilation time configuration. The section closes with
a few instruction that check the functionality of the installed Papillon
software module and its components.


\subsection{Requirements}

In order to compile Papillon you need some general development environment.
In most cases all components have already been installed on your system. For
each required component in the listing below an Internet reference is given
that allows free download of the component or an equivalent software.\\

{\bf Make tool: {\code make}}\\
You need a command that automates the compilation process.  You can use the
Solaris make {\code /usr/ccs/bin/make} from the {\code SUNWsprot} package or
install GNU make that is part of the Solaris Companion Software~\cite{SCS}
or available at Sunfreeware~\cite{SFW}.\\

{\bf C Compiler: {\code cc} or {\code gcc}}\\ You need a C compiler that
supports the generation of 64 bit objects.  You can use the Sun C
Compiler~\cite{SCC} which is part of the  {\code SUNWspro} package and
bundled with the Sun ONE Studio 7 Compiler  Collection or the GNU C
Compiler~\cite{GCC} version 3.x or above available at
Sunfreeware~\cite{SFW}.\\

If you choose to use the GNU C Compiler, check that you are using {\em
version 3.x or above} for the {\em correct} Solaris version. You can use
the command {\code gcc --version} to retrieve both information.\\

{\bf Linker: {\code ld}}\\ 
A linker is also necessary to link the compiled object files. You can use
the default linker {\code /usr/ccs/bin/ld} from the {\code SUNWtoo} package
or the GNU linker which is {\em not} part of the Solaris Companion Software
but available at Sunfreeware~\cite{SFW}.

\subsection{Compilation Time Configuration}

Before building the Papillon module from the source files inside the  
{\code src/} directory. You should configure the features and protections
of the module to fit your needs. The following list shows files that contain
configurable parts.

\begin{center}
\begin{tabular}{ r | l } 
   {\code src/Makefile}  & Compilation details and pathnames \\
   {\code src/papillon.h} & User and group IDs, communication system call \\
   {\code src/papillon.c} & Default setting for features and protections \\
\end{tabular}
\end{center}

\subsubsection{Configuration {\tt src/Makefile}}

You need to decide which features and protections to compile into the module
and where to store the components of Papillon. By default all features and
protections are included. Edit the file {\code src/Makefile } and change the
following variables if necessary.\\

{\code SYSCONFDIR=/etc} \\ 
You should not change your system configuration directory unless you
store configuration and boot scripts in another directory, which is
very untypical.\\
   
{\code SBINDIR=/usr/sbin} \\
This is the location where the control tool {\code papctl} will be
installed. You may change this to any path as long as the Papillon module
and the {\code papctl} program stay on the same type of file system.\\
      
{\code KERNELDIR=/usr/kernel/misc }\\  
This is the place where the Papillon module will be installed.  There is no
need to change this unless you know what you are doing.\\

{\code FEATURES=[...] }\\ 
By changing the values of this variable you can exclude features. To exclude
a feature simply remove its definition from the {\code FEATURES} variable.
Excluded features are not compiled into the module, they  cannot be enabled
at later time without recompiling the module. Following is a list of all
possible definitions and the feature they enable.

\begin{center}
\begin{tabular}{ r | l } 
   {\code -DRSTPROC }   & Restricted Proc \\ 
   {\code -DSECSTDFD }  & Secure STDIO File Descriptors \\
   {\code -DPPROMISC }  & Pseudo Promiscuous Flag \\
   {\code -DMODHIDING } & Module Hiding \\ 
\end{tabular}
\end{center} 

{\code PROTECTIONS=[...] }\\ 
By changing the values of this variable you can exclude protections similar
to the {\code FEATURE} variable. To exclude a protection simply remove its
definition from the {\code PROTECTIONS} variable. As with the features,
excluded protection can only be included through recompilation. Following is 
a list of all possible definitions and the protection they enable. 

\begin{center}      
\begin{tabular}{  r | l}
    {\code -DSYMPROT }    & Symbolic Link Protection \\ 
    {\code -DFIFOPROT }   & FIFO Protection \\ 
    {\code -DHARDPROT }   & Hardlink Protection \\
    {\code -DCHROOTPROT } & Chroot Protection \\
    {\code -DSEXECPROT }  & Setuid Execution Protection \\
\end{tabular}
\end{center}     

{\code CC=gcc}\\ 
Depending on your C compiler you have to change this variable and the 
{\code CFLAGS32} and {\code CFLAGS64} variables. There are uncommented 
settings for both C compilers in the Makefile. If you are using the 
GNU C Compiler, you can also add {\code -Wall} to the {\code COPTS} 
variable in order to get all warning messages during the compilation 
process.\\

You may also modify other variables in the file {\code src/Makefile} but in
general everything should work without further modification on a default
installation of the Solaris Operating Environment.

\subsubsection{Configuration {\tt src/papillon.h}}

If you are an advanced user and have some experience with kernel modules,
you can also edit other files inside the {\code src/} directory. The
following changes can be done in {\code src/papillon.h}.

\setlength{\parskip}{\tocparskip}
\begin {itemize}
\item Changing the super-group\\
      If you want to grant read-access to a group of users inside the
      Restricted Proc, change the definition {\code SUSER\_GID} to an 
      existing Unix  group. By default read-access is granted to the 
      super-user group {\code GID 0}.

\item Changing the super-user\\ 
      Papillon associates the {\code UID 0} with the super-user. If for some
      reason you want to change this and also restrict {\code UID 0}, change
      the definition {\code SUSER\_UID} to a different user ID.

\item Changing the communication system call \\ 
      Papillon uses an unused system call for communication. {\code papctl}
      uses this system call to export and import the configuration of
      Papillon from user space to kernel space and vice versa.  The system
      call number is defined by {\code SYS\_papcomm}. If you are sure that
      this system call is used on your system, e.g. by a third party
      software, change the value to another unused system call. You can
      retrieve a list of all used system calls by examining the system
      header file {\code /usr/include/sys/systm.h}.
\end{itemize}
\setlength{\parskip}{\myparskip}

\subsubsection{Configuration {\tt src/papillon.c}}

If you are really experienced with C source code, you can also configure 
some settings in the file {\code src/papillon.c}. \\

When the Papillon module is loaded it activates a default configuration,
which can be changed at runtime using the tool {\code papctl}. To change this 
default configuration in {\code src/papillon.c}, modify the initial values of
the {\code pap\_config\_t config} struct. 

\bcode
\begin{verbatim}
pap_config_t config = {
    /* rstproc, ppromisc, modhiding, secstdfd */
       PAP_ON, PAP_ON, PAP_OFF, PAP_ON,
    /* fifoprot, symprot, hardprot, chrootprot, sexecprot */
       PAP_DENY, PAP_DENY, PAP_DENY, PAP_DENY, PAP_DENY
};
\end{verbatim}
\ecode

The Setuid Execution Protection can be trimmed using a white-list of
programs that don't pass the protection code. This list is specified inside
{\code src/papillon.c}. You can add new programs to the list as long as you
remember to end the list with a NULL pointer.

\bcode
\begin{verbatim}
const char *pap_whitelist[] = {
    "/usr/dt/bin/dtpower",
    NULL
};
\end{verbatim}
\ecode

The struct {\code pap\_modfiles\_t modfiles[]} holds the files to be hidden.
If you want to add a file, e.g. {\code /usr/bin/foobar},  extend the struct
by adding a new line before the the last triple NULL line. Note that you
can only hide files on the same file system type. 

\bcode
\begin{verbatim}
pap_modfiles_t modfiles[] = {
    [...]
    { "/usr/bin/foobar", NULL, NULL },
    {NULL, NULL, NULL}
};
\end{verbatim}
\ecode


\subsection{Compilation}

The compilation process itself is rather simple and straight-forward and
should build the module in a few seconds.

\bcode
\begin{verbatim}
# cd src 
# make 
# cd .. 
\end{verbatim}
\ecode

If an error occurs, check if you configured everything correctly as
described in the previous section. If you configured parts inside the  files
{\code papillon.h} and {\code papillon.c} watch for typos, missing brackets
and semicolons.\\ 

If you still cannot build the module, pipe the output of the make process
into a file and send it to the author. See the section \ref{fb} for more
information about how to create a bug report and where to send feedback.

\subsection{Testing the Build}

Even though Papillon has been designed with stability and compatibility in
mind, it is wise to run some functionality and stability tests before
installing the module permanently. As the first step load the module into
the running kernel by executing the following command.

\bcode
\begin{verbatim}
# modload ./src/papillon 
\end{verbatim}
\ecode

If the execution of the program {\code modload} fails, it will report the
message {\em "No such file or directory"} even though all files are in
place. Don't get confused, consult the syslog for information about the
failure, if necessary extend the syslog logging configuration to catch 
kernel error events. If no failure is reported, run the control tool 
{\code papctl} to check if the module has been loaded successfully and the
configuration gets exported.

\bcode
\begin{verbatim}
# ./src/papctl -g 
Current configuration of the Papillon v0.5.0 module:
[...]
\end{verbatim}
\ecode

In order to allow functionality tests a test suite has been added to  the
Papillon source package that consists of several C sources and  a shell
script that helps generating the test environment. Compile the test suite by
executing the commands below. Maybe it is necessary to make some  changes in
the corresponding Makefile {\code test/Makefile}.

\bcode
\begin{verbatim}
# cd test
# make
\end{verbatim}
\ecode

Stay in the {\code test/} directory and execute the shell script {\code
test.sh} that will build a test environment and launch several fake attacks
to test the features and protections of the Papillon module. All of these
fake attacks have been designed with security in mind and  minimal
privileges by using the {\em nobody user} as an attacker, nevertheless you
should shutdown critical system services and disable logins during the
execution of the test suite.

\bcode
\begin{verbatim}
# ./test.sh
[...]
* General environment
- Checking for a restricted proc...           Yes
- Checking for hardlink attack protection...  Yes
- Checking for symlink attack protection...   Yes
[...]
Done.
\end{verbatim}
\ecode

The output of the test shell script may vary depending on the compilation
time configuration you have made. If you enable all protections and features
the output should show a {\em yes} at the end of each fake attack.\\

In order to complete the tests, make the module visible and unload it. Use
{\code modinfo} to determine the module ID of Papillon and replace 
{\code ID} in the last line with this number.

\bcode
\begin{verbatim}
# cd ..
# ./src/papctl -s m=off
# modinfo
# modunload -i ID
\end{verbatim}
\ecode

If during the process described above the system panics or any other minor
or major problems occur, please spend some time and create a bug report. See
the section \ref{fb} for more information about how to create a bug report
and where to send feedback.

\subsubsection{Installation}

In order to install the Papillon kernel module and its components execute
the following sequence of programs that will install and launch Papillon. If
you have skipped the previous section that introduced the test suite for
Papillon, go back and perform the tests in order to guarantee system's
stability and Papillon's functionality.

\bcode
\begin{verbatim}
# cd src 
# make install 
# /etc/init.d/papillon start 
\end{verbatim}
\ecode

These commands will create the following files on your system and enable 
Papillon automatically the next time you reboot your system. If you have
modified some of the path variables in the file {\code src/Makefile},
pathnames may differ.

\begin{center}
\begin{tabular}{ >{\code}r | l}
/usr/kernel/misc/papillon & The 32 bit kernel module \\
/usr/kernel/misc/sparcv9/papillon & The 64 bit kernel module \\
/usr/sbin/papctl & The control tool\\
/etc/init.d/papillon & The init script to load papillon \\
/etc/rc*.d/*papillon & The hard links to the init script \\
\end{tabular}
\end{center}

If you don't like Papillon, you can use the following sequence of programs
to uninstall the module and the corresponding files. Note that is necessary
to unload the module from the kernel before removing the file when module
hiding is activated.

\bcode
\begin{verbatim}
# make uninstall 
\end{verbatim}
\ecode

If you don't like Papillon, why not drop a note to the author about the 
things you dislike? See section \ref{fb} for contact information.

\section{Runtime Configuration}

If Papillon is loaded, you can use the control tool {\code papctl} to toggle
features and protections. Below is a list of the command line options  and
some examples. \\

Note that if the Papillon module is loaded and hidden, you are not able to
view it on the list of loaded modules generated by {\code modinfo}. The
control tool {\code papctl} is the only way to test if the module is loaded.

\subsection{Control Tool Options}

This section covers the command line options of the control tool {\code 
papctl} which is available with the Papillon module and used to communicate
from user land with the kernel module.

\begin {tabular} { r l l}
Usage:   & \\ 
         & \multicolumn{2}{l}{\code papctl [-fhV] -g | -s variable=value [variable=value ...]} \\ 
Options: &           & \\ 
         & {\code -g } & get current configuration of the loaded module. \\ 
         & {\code -s variable=value ...} 
         & set current configuration of the loaded module.\\ 
	 & {\code -f} & force setting the current configuration. \\
         & {\code -h} & print this help.\\
         & {\code -V} & print version information.\\
\end{tabular} \\ 

In order to toggle features or protections you have to assign variables the
corresponding values. This is the table of all variables, their values and
their description.

\begin{center}
\begin{tabular} {  c  l  l  } 
  \textbf{Variable} & \textbf{Feature Description} & \textbf{Possible values} \\ 
  \hline 
  {\code r } & Restricted Proc & {\code on, off } \\ 
  {\code p } & Pseudo Promiscuous Flag & {\code on, off } \\ 
  {\code m } & Module Hiding & {\code on, off} \\ 
  {\code i } & Secure STDIO File Descriptor & {\code on, off } \\ 
\\
  \textbf{Variable} & \textbf{Protection Description} & \textbf{Possible values} \\ 
  \hline 
  {\code s } & Symbolic Link Protection & {\code none, warn, deny } \\ 
  {\code h } & Hardlink Protection & {\code none, warn, deny } \\ 
  {\code f } & Fifo Protection & {\code none, warn, deny } \\ 
  {\code c } & Chroot Protection & {\code none, warn, deny } \\
  {\code x } & Setuid Execution Protection & {\code none, warn, deny } \\
\end{tabular}
\end{center}

\subsection {Command Line Examples} 

\begin {enumerate} 
\item One of the most common situations is the manual unloading of the
      module. In general this can be done by using the init script, but
      for completeness the following example demonstrates how to turn 
      of the module hiding and then unload the module.
      
\bcode
\begin{verbatim}
# papctl -s m=off
# ID=`modinfo | grep papillon | cut -d1 -f" "` 
# modunload -i $ID
\end{verbatim}
\bcode

\item If you want to disable the complete module but not unload it, you have
      to disable all features and protections except module hiding. The
      module will then simply idle and wait to be activated again. The
      following example illustrates the long command line for this purpose.
      Note that the module is not visible in this example ({\code m=on}).

\bcode
\begin{verbatim}
# papctl -s r=off p=off m=on i=off \
            s=none h=none f=none c=none x=none
\end{verbatim}
\ecode

\item If you want to view the current configuration and then enable all
      features of Papillon and leave the protections' configuration
      untouched, execute the following sequence of commands.
\bcode
\begin{verbatim}
# papctl -g
# papctl -s r=on p=on m=on i=on 
\end{verbatim}
\ecode

\item This last example demonstrates some of the fake attacks that are 
      performed in the Papillon test suite. If you have enabled the 
      corresponding features and protections you should notice the 
      enhanced security. 
      
\bcode
\begin{verbatim}
# touch /tmp/a
# su nobody -c "ln -s /tmp/a /tmp/b"

# /etc/init.d/papillon stop
# su nobody -c "ln /tmp/a /tmp/c"
# su nobody -c "ps -e"
# cat /tmp/b
# rm /tmp/c

# /etc/init./papillon start
# su nobody -c "ln /tmp/a /tmp/c"
# su nobody -c "ps -e"
# cat /tmp/b
# rm /tmp/a /tmp/b /tmp/c
\end{verbatim}
\ecode
     

\end{enumerate}

\chapter{Closing Words}

\section{Known Problems}

\subsubsection{Control tool blocks}

The Papillon kernel module uses strict locking mechanisms to prevent  data
inconsistency therefore the internal configuration is protected by a 
read-write-lock. This lock allows concurrent execution of the control 
tool {\code papctl}.\\ 

If a kernel thread operates inside a system call that is intercepted by the
module, the control tool {\code papctl} might return the following error
message and doesn't update the configuration.

\bcode
\begin{verbatim}
# papctl -s  m=off
Error 16 
Configuration blocked.
\end{verbatim}
\ecode

Usually this is a temporary problem and waiting a few seconds is a possible
solution. If the control tool still blocks, some application's kernel thread
is blocked within an intercepted system call. The best solution is to find
this application and terminate it, in most cases leaving the graphical
interface solves the problem. Unfortunately finding the application can be
an annoying challenge. You can use the {\code -f} option to force setting
the configuration, but you will risk inconsistency in Papillon's
configuration.

\bcode
\begin{verbatim}
# papctl -f -s m=off
\end{verbatim}
\ecode

\subsubsection{System panics on Shutdown}

The module hiding feature may cause a kernel panic on shutdown, if parts of
Papillon are located on a network file system or if the system uses third
party file systems as AFS.\\

As a temporary solution recompile Papillon with disabled module hiding. If 
you still experience kernel panics on shutdown, contact the author of 
Papillon.

\subsubsection{Lost Inodes}

Some people have reported lost inodes if rebooting the system after the 
Papillon kernel module has been loaded. The inodes loose their reference if
the module is loaded {\em twice} into the kernel.\\

Avoid loading the kernel module twice. Before loading the module manually
into kernel, check if the module has been already loaded at boot time and
unload it if necessary. You can find out if the module has been loaded using
the control tool. 

\bcode
\begin{verbatim}
# papctl -g
Error 12
Papillon is not loaded. 
\end{verbatim}
\ecode

\section{Feedback}
\label{fb}
Papillon is not yet another security solution by a major company, it is an 
non-commercial open source project aiming at security enhancing the Solaris
Operating Environment.  The Papillon module, its components and the
documentation have been written in the free time of the author, therefore
your feedback is {\em essential}.\\

If you discover a bug, the systems panics, you can't compile the source or
there is anything else you like to comment, please feel free to drop a mail
to Konrad Rieck {\code (kr@roqe.org)}. If you are reporting a problem with
Papillon include information about your actual system setup, e.g. by
executing the following commands. 

\bcode
\begin{verbatim}
# uname -a 
# isainfo -v -b
# psrinfo -v
# dmesg | tail -10
\end{verbatim}
\ecode

If you want to report a compilation problem pipe the output of the 
make process into a file and attach it to your mail.

\bcode
\begin{verbatim}
# make > /tmp/report 2>&1
\end{verbatim}
\ecode

If the system panics and dumps core, follow the instructions below to
generate a stack back-trace of the core image and include the trace in the
email. Please only send these traces and don't send core images via email, 
because they are very large.

\bcode
\begin{verbatim}
# cd /var/crash/`hostname`
# echo \$c | mdb unix.0 vmcore.0 
\end{verbatim}
\ecode

If you have ideas or criticism regarding Papillon and its functionality feel
free to drop an email and if you are an experienced programmer take
a look Papillon's source, maybe you can contribute a new feature or
security protection.\\

If you are using Papillon in a company or large network and feel that it is
great software, please send an email or, if you like, a little donation.

\newpage
\section{Thanks}

The author would like to thank the following people (in no special order):

\begin{tabular}{p{3,5cm} p{12cm}}

--~~ Job de Haas & For his ideas, support and the fun at HAL2001 conference \\

--~~ Heiko Krupp & For contributing the initial implementation of the chroot
                 protection.\\

--~~ Casper Dik & For his helpful advices on the system-dependent GCC
                headers and mysterious bug 1170077.\\

\end{tabular}\\[5mm]

Thanks to all the people who sent in bug reports and feedback (in order of
"appearance"):
\begin{multicols}{3}
   Philipp Stucke \\
   Fabian Kr�nner \\
   Sergei~Rousakov\\
   Michael~Parkin\\
   Adam~Mazza\\ 
   Adam~Morley\\ 
   Juri~Haberland\\
   Erik~Parker\\
   Eric~Thern\\ 
   Rikard~Skjelsvik\\
   Ray~Stirbei\\
   Martin~Preen\\
   Cedric~Marsot\\
   Leon~Halford\\
   John~Cartwright\\
   Marcin~Glowacki\\
   Jo{\~ao}~Miguel~Chaves\\
   Jukka~A.~Ukkonen\\
   Gianpaolo~Fasoli\\
   Jakub Wartak\\
\end{multicols}

\setlength{\parskip}{\tocparskip}
\nocite{*}
\begin{raggedright}
\bibliographystyle{mysiam}
\bibliography{refs}
\end{raggedright}

\appendix
\chapter{Appendix}
\section{{\tt CHANGES}}
\footnotesize
\verbatiminput{../CHANGES}
\section{{\tt LICENSE}}
\footnotesize
\verbatiminput{../LICENSE}

\end{document}

